%%%%%%%%%%%%%%%%%%%%%%%%%%%%%%%%%%%%%%%%%
% Beamer Presentation
% LaTeX Template
% Version 1.0 (10/11/12)
%
% This template has been downloaded from: 
% http: //www.LaTeXTemplates.com
%
% License: 
% CC BY-NC-SA 3.0 (http: //creativecommons.org/licenses/by-nc-sa/3.0/)
%
%%%%%%%%%%%%%%%%%%%%%%%%%%%%%%%%%%%%%%%%%

%----------------------------------------------------------------------------------------
%	PACKAGES AND THEMES
%----------------------------------------------------------------------------------------
%!TEX program = xelatex
\documentclass{beamer}

\mode<presentation> {

% The Beamer class comes with a number of default slide themes
% which change the colors and layouts of slides. Below this is a list
% of all the themes, uncomment each in turn to see what they look like.

%\usetheme{default}
%\usetheme{AnnArbor}
%\usetheme{Antibes}
%\usetheme{Bergen}
%\usetheme{Berkeley}
%\usetheme{Berlin}
%\usetheme{Boadilla}
%\usetheme{CambridgeUS}
%\usetheme{Copenhagen}
%\usetheme{Darmstadt}
%\usetheme{Dresden}
%\usetheme{Frankfurt}
%\usetheme{Goettingen}
%\usetheme{Hannover}
%\usetheme{Ilmenau}
%\usetheme{JuanLesPins}
%\usetheme{Luebeck}
\usetheme{Madrid}
%\usetheme{Malmoe}
%\usetheme{Marburg}
%\usetheme{Montpellier}
%\usetheme{PaloAlto}
%\usetheme{Pittsburgh}
%\usetheme{Rochester}
%\usetheme{Singapore}
%\usetheme{Szeged}
%\usetheme{Warsaw}

% As well as themes, the Beamer class has a number of color themes
% for any slide theme. Uncomment each of these in turn to see how it
% changes the colors of your current slide theme.

%\usecolortheme{albatross}
%\usecolortheme{beaver}
%\usecolortheme{beetle}
%\usecolortheme{crane}
%\usecolortheme{dolphin}
%\usecolortheme{dove}
%\usecolortheme{fly}
%\usecolortheme{lily}
%\usecolortheme{orchid}
%\usecolortheme{rose}
%\usecolortheme{seagull}
%\usecolortheme{seahorse}
\usecolortheme{whale}
%\usecolortheme{wolverine}

%\setbeamertemplate{footline} % To remove the footer line in all slides uncomment this line
%\setbeamertemplate{footline}[page number] % To replace the footer line in all slides with a simple slide count uncomment this line

%\setbeamertemplate{navigation symbols}{} % To remove the navigation symbols from the bottom of all slides uncomment this line
}

\usepackage{graphicx} % Allows including images
\graphicspath{{img/}}
\usepackage{booktabs} % Allows the use of \toprule, \midrule and \bottomrule in tables
%\usepackage[UTF8]{ctex}
\usefonttheme{professionalfonts}
\usepackage{amsmath}
\usepackage{amssymb}
\usepackage{subfigure}
%\usepackage{txfont}
\usepackage{xeCJK}
\usepackage[UTF8]{ctex}

\usepackage{fontspec}
%\setCJKmainfont{SimSun} %或\setCJKmainfont{KaiTi}
%\setCJKmonofont{SimSun}
%\newcommand{\tnewroman}{\fontspec{Times New Roman}}
\setmainfont{Times New Roman}
\usepackage{bm}
\usepackage{wrapfig}
\usepackage{movie15}
\usepackage{media9}
%\usepackage[dvipdfmx]{movie15_dvipdfmx}
%\usepackage{listings}


%----------------------------------------------------------------------------------------
%	TITLE PAGE
%----------------------------------------------------------------------------------------

\title[复变函数]{\Huge{复变函数与积分变换 }} % The short title appears at the bottom of every slide, the full title is only on the title page

\author{马越} % Your name
\institute[北京理工大学机械与车辆学院 ] % Your institution as it will appear on the bottom of every slide, may be shorthand to save space
{
北京理工大学 \\ % Your institution for the title page
\medskip
\textbf{yuema.bit@gmail.com} % Your email address
}
\date{\today} % Date, can be changed to a custom date


\boldmath


\begin{document}


%%%%%%%%%%%%%%%%%%%%%%%%%%%%%%%%%%%%%%%%%%%%%%%%%%%%
%%%%%%%%%%%%%%%%%%%%%%%%%%%%%%%%%%%%%%%%%%%%%%%%%%%%

%page 1

%%%%%%%%%%%%%%%%%%%%%%%%%%%%%%%%%%%%%%%%%%%%%%%%%%%%
%%%%%%%%%%%%%%%%%%%%%%%%%%%%%%%%%%%%%%%%%%%%%%%%%%%%


\begin{frame}[t]
\titlepage % Print the title page as the first slide
\end{frame}

\begin{frame}[t]
\frametitle{个人简介}
\begin{itemize}
    \item 姓    名:马越
    \item 工作单位:三院车辆工程系特种车辆研究所
\item 职    称:副教授
\item 联系方式:\\
手机:18618194096\\
座机:68918489     \\  
邮箱:armcynicism@bit.edu.cn, yuema.bit@gmail.com
\item 办公地点:@9\#435
\item 研究方向:智能车辆的动力学及其控制,混合动力车辆控制,计算智能及边缘计算
\end{itemize}

\end{frame}

%----------------------------------------------------------------------------------------
%	PRESENTATION SLIDES
%----------------------------------------------------------------------------------------

%------------------------------------------------

%%%%%%%%%%%%%%%%%%%%%%%%%%%%%%%%%%%%%%%%%%%%%%%%%%%%
%%%%%%%%%%%%%%%%%%%%%%%%%%%%%%%%%%%%%%%%%%%%%%%%%%%%

%page 2

%%%%%%%%%%%%%%%%%%%%%%%%%%%%%%%%%%%%%%%%%%%%%%%%%%%%
%%%%%%%%%%%%%%%%%%%%%%%%%%%%%%%%%%%%%%%%%%%%%%%%%%%%
\begin{frame}[t]
\frametitle{课程简介}
% Sections can be created in order to organize your presentation into discrete blocks, all sections and subsections are automatically printed in the table of contents as an overview of the talk
%------------------------------------------------
\begin{block}{课程组成}
由复变函数 (1-5章) 和积分变换 (第6章) 两部分组成
\end{block}

\begin{block}{教材}
包革军, 邢宇明, 盖云英等, 复变函数与积分变换(第三版), 科学出版社
\end{block}

\begin{block}{参考书}
1.詹姆斯•沃德•布朗 (作者), 张继龙(译者), 复变函数及其应用, 机械工业出版社\\
2. Tristan Needham, 复分析: 可视化方法, 人民邮电出版社\\
3. 龚昇, 简明复分析, 中国科技大学出版社
\end{block}

\end{frame}
%%%%%%%%%%%%%%%%%%%%%%%%%%%%%%%%%%%%%%%%%%%%%%%%%%%%
%%%%%%%%%%%%%%%%%%%%%%%%%%%%%%%%%%%%%%%%%%%%%%%%%%%%

%page 3

%%%%%%%%%%%%%%%%%%%%%%%%%%%%%%%%%%%%%%%%%%%%%%%%%%%%
%%%%%%%%%%%%%%%%%%%%%%%%%%%%%%%%%%%%%%%%%%%%%%%%%%%%
\begin{frame}[t]
\frametitle{课程简介}

\begin{table}
\begin{tabular}{l p{7cm} }
\toprule
对象 & 复变函数(自变量为复数的函数) \\
\midrule
主要任务 & 研究复变数之间的相互关系
主要是复数域上的微积分 \\
\midrule
主要内容 & 复数与复变函数、解析函数、
复变函数的积分、级数、留数、
共形映射、傅立叶变换和拉普拉斯变换等 \\
\bottomrule
\end{tabular}
%\caption{Table caption}
\end{table}

\begin{block}{学习方法}
\begin{itemize}
\item 高等数学中的概念在复数域中的推广. 注意相似点和不同点, 更要注意推广方法
\item 用大学的学习方法学习, 即注重概念、方法的学习, 重视定理的证明
\item 将大学学到的知识和方法尽可能用到本课的学习
\end{itemize}
\end{block}
\textit{http://u.163.com/n3MF3hbW  提取码: zJrMHXgf}

\end{frame}

%%%%%%%%%%%%%%%%%%%%%%%%%%%%%%%%%%%%%%%%%%%%%%%%%%%%
%%%%%%%%%%%%%%%%%%%%%%%%%%%%%%%%%%%%%%%%%%%%%%%%%%%%

%page 4

%%%%%%%%%%%%%%%%%%%%%%%%%%%%%%%%%%%%%%%%%%%%%%%%%%%%
%%%%%%%%%%%%%%%%%%%%%%%%%%%%%%%%%%%%%%%%%%%%%%%%%%%%

\begin{frame}[t]
\frametitle{引言}
\begin{block}{世界著名数学家 M.Kline指出:\footnote{\color{white}莫里斯 \quad 克莱因. 古今数学思想. \quad 上海科学技术出版社}}
\begin{itemize}
\item 19世纪最独特的创造是复变函数理论:象微积分的直接扩展统治了18世纪那样,该数学分支几乎统治了19世纪。
\item
    曾被称为这个世纪的数学享受,也曾作为抽象科学中最和谐的理论。
\end{itemize}

\end{block}

\begin{itemize}
\item
    16世纪,在解代数方程时引进复数:\\
     1545年,意大利Cardan在方程$x(10-x) = 40$时,首先产生了负数开平方的思想。方程的根:$x = 5 \pm \sqrt{-15}$.两个根的和是10,乘积是40。
\item
为使负数开平方有意义,引入了虚数,使实数域扩大到复数域。Bombelli、Leibniz等都利用了复数。
\item
在18世纪以前,对复数的概念及性质了解得不清楚,用它们进行计算又得到一些矛盾.人们一直把复数看作不能接受的“虚数”-笛卡尔

\end{itemize}
\end{frame}
%
%%%%%%%%%%%%%%%%%%%%%%%%%%%%%%%%%%%%%%%%%%%%%%%%%%%%%
%%%%%%%%%%%%%%%%%%%%%%%%%%%%%%%%%%%%%%%%%%%%%%%%%%%%%
%
%%page 5
%
%%%%%%%%%%%%%%%%%%%%%%%%%%%%%%%%%%%%%%%%%%%%%%%%%%%%%
%%%%%%%%%%%%%%%%%%%%%%%%%%%%%%%%%%%%%%%%%%%%%%%%%%%%%
%
\begin{frame}{引言}

\begin{itemize}
    \item 
      直到18世纪,J.D’Alembert(1717-1783)与L.Euler(1707-1783)等人建立了复数的几何解释。
逐步阐明了复数的几何意义和物理意义,澄清了复数的概念    
     \item
        Euler系统地建立了复数理论,创建了复变函数论的一些基本定理,并首创用符号“i”作为虚数的单位。
     \item
     应用复数和复变函数研究了流体力学等方面的一些问题.复数被广泛承认接受,复变函数论顺利建立和发展.

\end{itemize}
    到19世纪,完全奠定了复变函数的理论基础
三位代表人物:
\begin{itemize}
    \item A.L.Cauchy (1789-1866):应用积分方法研究复变函数
 \item
 K.Weierstrass(1815-1897)应用级数方法研究复变函数
 \item
 G.F.B.Riemann (1826-1866)应用几何方法研究复变函数
\end{itemize}
 通过他们的努力,复变函数形成了系统的理论,且渗透到了数学的许多分支;同时,它在热力学,流体力学、电学和经典控制理论等方面也得到了广泛的应用.

\end{frame}
%%%%%%%%%%%%%%%%%%%%%%%%%%%%%%%%%%%%%%%%%%%%%%%%%%%%%
%%%%%%%%%%%%%%%%%%%%%%%%%%%%%%%%%%%%%%%%%%%%%%%%%%%%%
%
%%page 6
%
%%%%%%%%%%%%%%%%%%%%%%%%%%%%%%%%%%%%%%%%%%%%%%%%%%%%%
%%%%%%%%%%%%%%%%%%%%%%%%%%%%%%%%%%%%%%%%%%%%%%%%%%%%%
\begin{frame}{引言}
\textbf{复变函数的基本思想 \footnote{ 埃伯哈德 .蔡德勒\quad 数学指南-实用数学手册 \quad 科学出版社}}
\begin{enumerate}
    \item 开集上的每个可微复值函数都能局部地展开成幂级数:即它是解析的:比无穷次光滑可导还好,可以局部展开为收敛的幂级数。
    \item  单连通区域中的解析函数的周线积分与积分路径无关
    \item 在一个点的某个解析的局部给定的每个函数,可以唯一扩充成整体解析的函数。
\end{enumerate}
\centering{\fbox{ \color{red}一个解析函数的局部行为唯一决定了它的整体行为。}}
\begin{itemize}
    \item 用自然的方式解释了实分析中的很多生硬问题。
    \item 有简洁的结构,积分性质完全由奇点性质决定。这一性质对计算的简化也是非常可观的-数学是避免计算的艺术-民间传说
    \item 有十分漂亮的几何直观,与拓扑学联系紧密。
\end{itemize}
\uwave{如果说实变函数研究的是怪物的话,那复变函数研究的就是尤物。-知乎某网友}

\end{frame}

\begin{frame}
\frametitle{引言} 
\textbf{复变函数之美}
\begin{figure}[H]
\centering  %图片全局居中
\subfigure[]{
\includegraphics[width=0.3\textwidth]{01}}
\subfigure[]{
\includegraphics[width=0.3\textwidth]{02}}
\subfigure[]{
\includegraphics[width=0.3\textwidth]{03}}
\subfigure[]{
\includegraphics[width=0.3\textwidth]{05}}
\subfigure[]{
\includegraphics[width=0.3\textwidth]{06}}
\subfigure[]{
\includegraphics[width=0.3\textwidth]{04}}
\end{figure}


\end{frame}

\begin{frame}{引言}
\textbf{复变函数的应用}
\begin{itemize}
    

    \item 数学和物理领域\\ 
    \begin{itemize}
    \item 解代数方程。如
        $x^2+1=0$    
    在实数范围内无解,引入复数则可以得到解。 \\
    Gauss用复数理论证明了代数基本定理。 
    \item 积分的计算,如$\int_0^\infty{\frac{\sin{x}}{x(x^2+1)}dx}$
    \item 求解偏微分方程,如 
        $$\frac{\partial^2{u}}{\partial{x^2}} + \frac{\partial^2{u}}{\partial{y^2}} =0 $$    
    具体应用在平面稳定流动(水、空气的流速与几乎时间无关)
    \end{itemize}


    \item 工程领域:\\ 
    \begin{itemize} 
    \item 应用于计算绕流问题中的压力、力矩:飞机机翼剖面压力的计算,解决机翼造型问题。
    \item 计算渗流问题。\\
      例如:大坝、钻井的浸润曲线。
    \item 平面热传导问题、电(磁)场强度。\\
      例如:热炉中温度的计算。
    \end{itemize}
    
\end{itemize}

\end{frame}

\begin{frame}{引言}
\textbf{复变函数的应用(续)}
\begin{itemize}
    \item 新能源汽车:无线充电\\
    
\begin{figure}
    \setcounter{subfigure}{0}
\centering  %图片全局居中
\subfigure[]{
\includegraphics[width=0.5\textwidth]{WirelessCharging01.png}}
%\subfigure[]{
%\includegraphics[width=0.3\textwidth]{WirelessCharging02.png}}
\subfigure[]{
\includegraphics[width=0.25\textwidth]{WirelessCharging03.png}}
\subfigure[]{
\includegraphics[width=0.2\textwidth]{WirelessCharging04.png}}
\subfigure[]{
\includegraphics[width=0.3\textwidth]{WirelessCharging05.png}}

\end{figure}

    
\end{itemize}



\end{frame}



\begin{frame}{引言}
\textbf{复变函数的应用(续)}
\begin{itemize}

    \item 自动控制理论
    
    \begin{itemize}
        \item 传递函数:Laplace变换\\
        常微分方程:
        \begin{equation}
        \ddot{x}+3\dot{x}+2x=H(t), \ \  x(0)=\dot{x}(0) =0. 
\end{equation}
        其传递函数的Laplace变换:
        $$\frac{X(s)}{U(s)}=\frac{1}{s}\left(\frac{1}{(s+2)(s+1)}\right)$$
        \item 稳定性判据:幅角定理
        \begin{figure}
        \setcounter{figure}{0}
        \includegraphics[width=0.4\textwidth]{Amp_Angle}
        \end{figure}
    \end{itemize}
     \end{itemize}
\end{frame}


%%%%%%%%%%%%%%%%%%%%%%%%%%%%%%%%%%%%%%%%%%%%%%%%%%%%%
%%%%%%%%%%%%%%%%%%%%%%%%%%%%%%%%%%%%%%%%%%%%%%%%%%%%%
%
%%page 7
%
%%%%%%%%%%%%%%%%%%%%%%%%%%%%%%%%%%%%%%%%%%%%%%%%%%%%%
%%%%%%%%%%%%%%%%%%%%%%%%%%%%%%%%%%%%%%%%%%%%%%%%%%%%


%
%
%
%%------------------------------------------------
%
\section{第一章 \quad 复数与复变函数}
%%%%%%%%%%%%%%%%%%%%%%%%%%%%%%%%%%%%%%%%%%%%%%%%%%%%%
%%%%%%%%%%%%%%%%%%%%%%%%%%%%%%%%%%%%%%%%%%%%%%%%%%%%%
%
%%page 8
%
%%%%%%%%%%%%%%%%%%%%%%%%%%%%%%%%%%%%%%%%%%%%%%%%%%%%%
%%%%%%%%%%%%%%%%%%%%%%%%%%%%%%%%%%%%%%%%%%%%%%%%%%%%%

\begin{frame}

\frametitle{}

\begin{block}{\begin{center}
  \Huge{第一章 \quad 复数与复变函数}
\end{center}}
本章主要引入复数的概念及其运算, 平面点集的概念, 复变函数的极限与连续性等概念. 
\end{block}


\end{frame}
%
%%%%%%%%%%%%%%%%%%%%%%%%%%%%%%%%%%%%%%%%%%%%%%%%%%%%%
%%%%%%%%%%%%%%%%%%%%%%%%%%%%%%%%%%%%%%%%%%%%%%%%%%%%%
%
%%page 9
%
%%%%%%%%%%%%%%%%%%%%%%%%%%%%%%%%%%%%%%%%%%%%%%%%%%%%%
%%%%%%%%%%%%%%%%%%%%%%%%%%%%%%%%%%%%%%%%%%%%%%%%%%%%%

\begin{frame}[t]
\frametitle{课程目录}
\tableofcontents % Throughout your presentation, if you choose to use \section{} and \subsection{} commands, these will automatically be printed on this slide as an overview of your presentation



\end{frame}

\subsection{复数运算及几何表示}

\begin{frame}[t]
\frametitle{课程目录}
\tableofcontents[currentsubsection ] % Throughout your presentation, if you choose to use \section{} and \subsection{} commands, these will automatically be printed on this slide as an overview of your presentation
\end{frame}





\subsubsection{复数的概念}
\begin{frame}[t]
\frametitle{课程目录}
\tableofcontents[currentsubsection ] % Throughout your presentation, if you choose to use \section{} and \subsection{} commands, these will automatically be printed on this slide as an overview of your presentation
\end{frame}

%%%%%%%%%%%%%%%%%%%%%%%%%%%%%%%%%%%%%%%%%%%%%%%%%%%%
%%%%%%%%%%%%%%%%%%%%%%%%%%%%%%%%%%%%%%%%%%%%%%%%%%%%

%page 10

%%%%%%%%%%%%%%%%%%%%%%%%%%%%%%%%%%%%%%%%%%%%%%%%%%%%
%%%%%%%%%%%%%%%%%%%%%%%%%%%%%%%%%%%%%%%%%%%%%%%%%%%%

\begin{frame}[t]
\frametitle{复数的概念}

为了解方程的需要,例如:方程$x^2+1=0$在实数范围内无解,人们引入了一个新数${i}$,称为虚数单位\\

对虚数单位$i$,作如下规定:

\begin{enumerate}[(1)]
    \item 
    $$ i^2 = -1$$
    \item
    $i$可与实数一起按同样的法则进行四则运算
\end{enumerate}
\textbf{虚数单位i的性质:  幂次}
\begin{table}
\begin{tabular}{c c}

 $ i^1 = i$   & $i^{4n} = 1$ \\ 
 $ i^2 = -1 $  & $  i^{4n+1} = i$ \\
 $ i^3 = i \cdot i^2 = -i$  & $ i^{4n+2} = -1$  \\
 $ i^4 = i^2 \cdot i^2 = 1$ & $ i^{4n+3} = -i$
  
\end{tabular}
\end{table}
...
\end{frame}


%%%%%%%%%%%%%%%%%%%%%%%%%%%%%%%%%%%%%%%%%%%%%%%%%%%%%
%%%%%%%%%%%%%%%%%%%%%%%%%%%%%%%%%%%%%%%%%%%%%%%%%%%%%
%
%%page 11
%
%%%%%%%%%%%%%%%%%%%%%%%%%%%%%%%%%%%%%%%%%%%%%%%%%%%%%
%%%%%%%%%%%%%%%%%%%%%%%%%%%%%%%%%%%%%%%%%%%%%%%%%%%%%
\begin{frame}[t]
\frametitle{复数的概念}

\begin{alertblock}{定义:复数}
$\forall x, y\in\mathbb{R}$,称$z =x + iy$为复数 
\begin{center}
    $x$ 称为$z$ 的实部,记为 $x = \mathbf{Re}[z]$\\
$y$ 称为$z$ 的虚部,记为 $y =\mathbf{Im}[z]$ \\
\end{center}
\end{alertblock}


当$x=0$,$y \neq 0$时,$z = iy$称为纯虚数;
当$y=0$时,$z = x + 0i$\\
实数是复数的一部分,复数是实数的扩展。 

\begin{center}
\alert{注意:$x$,$y$选取有顺序要求。} 
\end{center}


\begin{alertblock}{定义:共轭复数}
实部相同而虚部绝对值相等、符号相反的两个复数称为共轭复数。$z$的共轭复数记为$\bar{z}$。若$z= x+iy$,则$\bar{z} = x- iy$。

\end{alertblock}

\end{frame}
%%%%%%%%%%%%%%%%%%%%%%%%%%%%%%%%%%%%%%%%%%%%%%%%%%%%%
%%%%%%%%%%%%%%%%%%%%%%%%%%%%%%%%%%%%%%%%%%%%%%%%%%%%%
%
%%page 12
%
%%%%%%%%%%%%%%%%%%%%%%%%%%%%%%%%%%%%%%%%%%%%%%%%%%%%%
%%%%%%%%%%%%%%%%%%%%%%%%%%%%%%%%%%%%%%%%%%%%%%%%%%%%%
\subsubsection{复数的表示方法}

\begin{frame}[t]
\frametitle{课程目录}
\tableofcontents[currentsubsection ] % Throughout your presentation, if you choose to use \section{} and \subsection{} commands, these will automatically be printed on this slide as an overview of your presentation
\end{frame}

\begin{frame}[t]
\frametitle{复数的表示方法}
\begin{enumerate}
    \item 定义表示形式\\
用$x + iy$表示复数$z$,即$z = x + iy$
给定复数$z = x + iy$,则确定了实部$x$和虚部$y$;反之,给定实部$x$和虚部$y$,则完全确定了复数$z$-复数$z$与一对有序实数$(x,y)$构成了一一对应关系。\\
\alert{$x + iy$与$(x,y)$不加区别。}

\item 平面表示法(直角坐标表示法)\\

复数$z = x + iy$与平面上的点$(x,y)$一一对应。
\end{enumerate}

\begin{figure}
\centering
\includegraphics[scale=0.6]{22}
%\caption{Graph of $f(x)=ax^2|_{\{0.1, 0.3, 1.0, 3.0\}}$}
\end{figure}
用于表示复数的平面称为复平面。其横轴称为实轴,纵轴称为虚轴。复数 $z$ 是点$P$的同义词。    
\end{frame}


\begin{frame}[t]
\frametitle{复数的表示方法}
\begin{enumerate}
\setcounter{enumi}{2}
\item 复数的向量表示法\\
\begin{columns}
\column{0.7\textwidth}
平面上的点$P$与向量$OP$ 一一对应,因此复数$z$与向量$OP$ 一一对应。该向量的长度称为$z$的模或绝对值,记为\\
$$|z| = r =\sqrt{x^2 + y^2}$$
\column{0.3\textwidth}
\begin{figure}[htbp]
\includegraphics[scale=0.45]{23}
%\caption{Graph of $f(x)=ax^2|_{\{0.1, 0.3, 1.0, 3.0\}}$}
\end{figure}
\end{columns}

显然成立:
$$\mathbf{|x| \leq |z|, |y| \leq |z|},\mathbf{|z| \leq |x| + |y| ,  z \cdot \bar{z} = |z|^2 = |z^2|}
$$
\textbf{辐角的定义:}\\
当$z \neq 0$,时,把正实轴与向量OP的夹角称为$z$的辐角(argument),记为$Argz = \theta$\\
\alert{注意:任何一个复数$z \neq 0$有无穷多的辐角}
\\如果$\theta_1$是其中一个辐角,那么$z$的全部辐角为:\\
\begin{center}
$Argz = \theta_1 + 2k\pi $, $k$为任意整数
\end{center}

\alert{注意:当$z = 0$时,辐角不确定,没有辐角}

\end{enumerate}

\end{frame}

\begin{frame}[t]
\frametitle{复数的表示方法}
\begin{enumerate}
\setcounter{enumi}{2}
\item 复数的向量表示法\\

\textbf{辐角主值的定义:}\\
当$z$ ($z \neq 0$)的幅角中,把满足$-\pi < \theta_0 \leq \pi$的$\theta_0$称为$Argz$的主值,记作$\theta_0 = argz$
$$Argz = argz + 2k\pi \quad k=0, \pm1, \pm2 ...$$
\end{enumerate}
复数$z = x + iy (z \neq 0) $幅角主值:\\


\[ \begin{cases}
\arctan \frac{y}{x}, & x>0, \\
\pm \frac{\pi}{2}, & x=0, y \neq 0, \\
\arctan \frac{y}{x} \pm \pi, & x<0, y \neq 0, \\
\pi, & x< 0, y=0.
\end{cases} \]
其中$-\frac{\pi}{2}<\arctan{\frac{y}{x}}<\frac{\pi}{2}$
\end{frame}

\begin{frame}[t]
\frametitle{复数的表示方法}
\begin{enumerate}
\setcounter{enumi}{3}
\item 复数的三角表示法\\
利用直角坐标系与极坐标系的关系


\begin{equation}
 \left\{
\begin{array}{ll}
x & = r\cos{\theta}\\
y & = r\sin{\theta}

\end{array}
\nonumber
\right.
\end{equation}

复数可以表示成:
\begin{eqnarray}
z &=& x+iy \nonumber\\
&=& r(\cos{\theta}+i\sin{\theta})
\nonumber
\end{eqnarray}

\item 复数的指数表示法\\
利用Euler公式
$$e^{i\theta} = \cos{\theta}+i\sin{\theta}$$
则复数$r(\cos{\theta}+i\sin{\theta})$可以表示为:
$$z = re^{i\theta}$$
\end{enumerate}
\end{frame}

\begin{frame}
\frametitle{复数的表示方法}
\begin{exampleblock}{例 将下列复数化为三角表达式与指数表达式: \\ $(1)\quad z = -\sqrt{12} -2i; \quad(2) \quad z= \sin{\frac{\pi}{5}} + i \cos{\frac{\pi}{5}}$}
解:\\
(1)$ r = |z| =\sqrt{12+4} =4$,因$z$在第三象限\\
$\theta = \arctan{\frac{-2}{-\sqrt{12}}} - \pi = \arctan{\frac{\sqrt{3}}{3}}- \pi = -\frac{5\pi}{6}$.故:\\
$$z = 4 \left[ \cos{\left (-\frac{5\pi}{6}\right )} + i \sin{\left( -\frac{5\pi}{6} \right ) }  \right] = 4 e^{-\frac{5\pi}{6}i}$$
(2) 显然 $r = |z| =1$\\
$\sin{\frac{\pi}{5}} = \cos{ \left ( \frac{\pi}{2} - \frac{\pi}{5} \right) } = \cos{\frac{3\pi}{10}}$\\
$\cos{\frac{\pi}{5}} = \sin{ \left ( \frac{\pi}{2} - \frac{\pi}{5} \right) } = \sin{\frac{3\pi}{10}}$\\
故
$$z = \sin{\frac{\pi}{5}} + i \cos{\frac{\pi}{5}} = e^{\frac{3\pi}{10} i}$$
\end{exampleblock}
\end{frame}

\subsubsection{复数的代数运算}
\begin{frame}[t]
\frametitle{课程目录}
\tableofcontents[currentsubsection ] % Throughout your presentation, if you choose to use \section{} and \subsection{} commands, these will automatically be printed on this slide as an overview of your presentation
\end{frame}

\begin{frame}[t]
\frametitle{复数的代数运算}
\textbf{四则运算}
\begin{itemize}
\item{两个复数的相等}
\\复数$z_1,z_2$相等 $\Leftrightarrow$ 实部、虚部分别对应相等,即  $$z_1 = z_2 \Leftrightarrow \mathbf{Re}[z_1] = \mathbf{Re}[z_2],  \mathbf{Im}[z_1] = \mathbf{Im}[z_2]$$

\item{复数的大小:}
\\复数之间不能比较大小。
\\复数可以看成$i$ 的一次多项式。
\item 基本原则:复数的四则运算在实数要成立。
\item 复数的和与差
$$z_1\pm z_2 = (x_1 \pm x_2) + i (y_1 \pm y_2)$$
两个复数的和与差的运算类似多项式的和差运算,结果仍是复数

\end{itemize}
\end{frame}


\begin{frame}{复数的代数运算}
\textbf{复数的共轭}\\
根据前面的定义,实部相同, 虚部绝对值相等且符号相反的两个复数称为共轭复数.\\
 复数 $ z =x + iy $ 的共轭复数为 $ \bar{z} =x - i y $ 
\begin{block}{共轭复数的几何性质}
\begin{minipage}[b]{0.6\linewidth}
一对共轭复数 $ z $ 和 $ \bar{z} $ 在复平面的位置关于实轴对称\\\\
\end{minipage}\qquad
\begin{minipage}[t]{0.3\linewidth}
\includegraphics[width=\linewidth]{31}
\end{minipage}
\end{block}

\begin{exampleblock}{例:计算共轭复数 $ x+iy $ 与 $ x-iy $ 的积.}
解:
\[ (x+iy)(x-iy)=x^2-(iy)^2=x^2+y^2 \]
因此: 两个共轭复数 $ z $,  $ \bar{z} $  的积是一个实数.
\end{exampleblock}


\end{frame}




\begin{frame}{复数的代数运算}
\begin{block}{取共轭运算性质}
\begin{enumerate}[(1)]
	\item 共轭运算与加减乘除运算 
	\begin{eqnarray*}
	\overline {{z_1} \pm {z_2}} &=& {\bar z_1} \pm {\bar z_2} \\ \overline {{z_1} \cdot {z_2}} &=& {\bar z_1} \cdot {\bar z_2} \\
	 \overline {\left( {\frac{{{z_1}}}{{{z_2}}}} \right)} &=& \frac{{{{\bar z}_1}}}{{{{\bar z}_2}}}  
	\end{eqnarray*}
	
	即: 取共轭运算与加减乘除运算可交换
	
	\item 
	 $ \bar {\bar z }= z; $ 二次自反性. 即, $ z $ 与 $ \bar{z} $ 互为共轭
	\item 
	$ z \cdot \bar{z} = \left[ {\rm Re} (z) \right]^2 + \left[ {\rm Im} (z) \right]^2; $ 
	\item 
	$ z + \bar z = 2{\rm Re} (z), \quad z - \bar z = 2{\rm iIm} (z) $ 
	
	即利用取共轭运算可以由 $ z $ 计算其实部和虚部
	
\end{enumerate}
\end{block}
\end{frame}




\begin{frame}{复数的代数运算}
复数和的几何意义
\begin{figure}
\includegraphics[width=0.3\linewidth]{29a}
\end{figure}
复数差的几何意义
\begin{figure}
	\includegraphics[width=0.3\linewidth]{29b}
\end{figure}
\end{frame}
\begin{frame}{复数的代数运算}
\begin{block}{复数和与差的模的性质}
因为 $ \left| {{z_1} - {z_2}} \right| $ 表示点 $ {z_1} $ 和 $ {z_2} $ 之间的距离, 故
\begin{align*}
&\left| {{z_1} - {z_2}} \right| \ge \left| {\left| {{z_1}} \right| - \left| {{z_2}} \right|} \right|.\\
&\left| {{z_1} + {z_2}} \right| \le \left| {{z_1}} \right| + \left| {{z_2}} \right|.
\end{align*}
\begin{figure}
\includegraphics[width=0.3\linewidth]{30}
\end{figure}
\end{block}
\end{frame}

\begin{frame}[t]
\frametitle{复数的代数运算}
\textbf{四则运算}

\begin{itemize}

\item 复数的积
$$z_1 \cdot z_2 = (x_1x_2 - y_1y_2) + i (x_2y_1 +x_1y_2)$$
两个复数的积按多项式乘积运算,结果仍是复数。
\item 两个复数的商  $z_2 \neq  0$
\begin{eqnarray}
\frac{z_1}{z_2} &=& \frac{x_1x_2+y_1y_2}{x_2^2+y_2^2}+i\frac{x_2y_1-x_1y_2}{x_2^2+y_2^2} \nonumber \\
&=& \frac{z_1\cdot \bar{z}_2}{z_2 \cdot \bar{z}_2} \nonumber
\end{eqnarray}

\end{itemize}
\end{frame}

\begin{frame}
\frametitle{复数的代数运算}
\begin{exampleblock}{例: 设$z = -\frac{1}{i} - \frac{3i}{1-i}$,求$\mathbf{Re}(z), \mathbf{Im}(z)$与$z\cdot\bar{z}.$}
解: 
$$ z = -\frac{1}{i} - \frac{3i}{1-i} = -\frac{i}{i\cdot i} - \frac{3i(1+i)}{(1-i)(1+i)}= \frac{3}{2}-\frac{1}{2}i $$
所以:\\
$$\mathbf{Re}(z) = \frac{3}{2}, \mathbf{Im}(z) = -\frac{1}{2}$$\\
$$z\cdot\bar{z}  =  {\mathbf{Re}(z)}^2 + {\mathbf{Im}(z)}^2
= {\left (\frac{3}{2} \right )}^2+{\left (-\frac{1}{2}\right)}^2=\frac{5}{2}$$
\end{exampleblock} 

\end{frame}

\begin{frame}[t]
\frametitle{复数的代数运算}
\begin{exampleblock}{例: 设复数$z_1 =x_1 + iy_1, z_2 =x_2 + iy_2,$\\
证明:$z_1 \cdot \bar{z}_2 +  \bar{z}_1 \cdot z_2= 2 \mathbf {Re} (z_1 \cdot \bar{z}_2)$}
证: 
\begin{eqnarray*}
z_1 \cdot \bar{z}_2  + \bar{z}_1 \cdot z_2 \\
&=& (x_1 + iy_1)(x_2 - iy_2)+(x_1 - iy_1)(x_2 + iy_2)\\
&=& (x_1x_2+y_1y_2)+i(x_2y1-x_1y_2) \\
& & +(x_1x_2+y_1y_2)+i(-x_2y_1+x_1y_2)\\
&=& 2(x_1x_2+y_1y_2) = 2 \mathbf {Re} (z_1 \cdot \bar{z}_2)
\end{eqnarray*} \\

或:
 $$z_1 \cdot \bar{z}_2 + \bar{z}_1 \cdot z_2 = z_1 \cdot \bar{z}_2 + \overline{z_1 \cdot \bar{z}_2} =2 \mathbf {Re} (z_1 \cdot \bar{z}_2)$$
\end{exampleblock} 

\end{frame}



\begin{frame}[t]
\frametitle{复数的代数运算}
\textbf{乘幂与方根}\\
设复数 $ {z_1} $ 和 $ {z_2} $ 的三角形式分别为
\[ {z_1} = {r_1}(\cos {\theta _1} + i\sin {\theta _1}), {z_2} = {r_2}(\cos {\theta _2} + i\sin {\theta _2}), \]

则
\begin{align*}
{z_1} \cdot {z_2} &= {r_1}(\cos {\theta _1} + i\sin {\theta _1}) \cdot {r_2}(\cos {\theta _2} + i\sin {\theta _2})\\
&= {r_1} \cdot {r_2}[(\cos {\theta _1}\cos {\theta _2} - \sin {\theta _1}\sin {\theta _2}) + \\
& i(\sin {\theta _1}\cos {\theta _2} + \cos {\theta _1}\sin {\theta _2})]\\
& = {r_1} \cdot {r_2}[\cos ({\theta _1} + {\theta _2}) + i\sin ({\theta _1} + {\theta _2})]
\end{align*}
\[ \left| {\, {z_1}{z_2}\, } \right| = {r_1} \cdot {r_2} = \left| {{z_1}} \right| \cdot \left| {{z_2}} \right| \]
\[ {{\arg}}({z_1}{z_2}) = {{\arg}}{z_1} + {{\arg}}{z_2}. \]
\end{frame}

\begin{frame}[t]
\frametitle{复数的代数运算}
\textbf{乘幂与方根}\\
两个复数乘积的模等于它们的模的乘积; 两个复数乘积的辐角等于它们的辐角的和.

\begin{minipage}[b]{0.5\linewidth}
从几何上看, 两复数对应的向量分别为 $ {\vec z_1}, {\vec z_2} $,  先把 $ {{\vec z}_1} $ 按逆时针方向旋转一个角 $ {\theta _{2}} $,  再把它的模扩大到 $ {r_2} $ 倍, 所得向量 $ \vec z $ 就表示积 $ {z_1} \cdot {z_2} $. 
\end{minipage}\qquad
\begin{minipage}[b]{0.35\linewidth}
\includegraphics[width=\linewidth]{39}
\end{minipage}
\end{frame}

\begin{frame}[t]
\frametitle{复数的代数运算}
\textbf{乘幂与方根}\\
注意:  $ {\rm{Arg}}({z_1}{z_2}) = {\rm{Arg}}{z_1} + {\rm{Arg}}{z_2} $ 不是通常的等式, 

而是两个数集相等. 即左端任给一值,  右端必有值与它相对应; 反过来是如此.

例如: 设 $ {z_1} = - 1, {z_2} = i, $ 则 $ {z_1} \cdot {z_2} = - i, $ 
\begin{align*}
&{\rm{Arg}}{z_1} = \pi + 2n\pi, \quad (n = 0, \; \pm 1, \; \pm 2, \cdots ), \\
&{\rm{Arg}}{z_2} = \frac{\pi }{2} + 2m\pi, \quad (m = 0, \; \pm 1, \; \pm 2, \cdots ), \\
&{\rm{Arg(}}{z_1}{z_2}) = - \frac{\pi}{2} + 2k\pi, \quad (k = 0, \; \pm 1, \; \pm 2, \cdots ), 
\end{align*}
故 $ \frac{{3\pi }}{2} + 2(m + n)\pi = - \frac{\pi }{2} + 2k\pi $,  只须 $ k = m + n + 1. $ 
若 $ k = - 1 $,  则 $ m = 0, n = - 2 $ 或 $ m = - 2, n = 0 $. 
\end{frame}

\begin{frame}[t]
\frametitle{复数的代数运算}
\textbf{乘幂与方根}\\
设复数$z_1$和$z_2$的指数形式分别为
\[{z_1} = {r_1}{e^{i{\theta _1}}}, {z_2} = {r_2}{e^{i{\theta _2}}}, \]
则
\[{z_1} \cdot {z_2} = {r_1} \cdot {r_2}{e^{i({\theta _1} + {\theta _2})}}.\]
 $ n $ 个复数相乘的情况: 

设 $ {z_k} = {r_k}(\cos {\theta _k} + i\sin {\theta _k}) = {r_k}{e^{i{\theta _k}}}, \quad (k = 1, 2, \cdots, n) $ 
\begin{align*}
{z_1} \cdot {z_2} \cdot \cdots \cdot {z_n} = \prod\limits_{k = 1}^n {{z_k}} &= (\prod\limits_{k = 1}^n {{r_k})[\cos \sum\limits_{k = 1}^n {{\theta _k}} + i\sin \sum\limits_{k = 1}^n {{\theta _k}} ]} \\
& = \prod\limits_{k = 1}^n {{r_k}{e^{i\sum\limits_{k = 1}^n {{\theta _k}} }}}
\end{align*}
\end{frame}

\begin{frame}[t]
\frametitle{复数的代数运算}
\textbf{乘幂与方根}\\
同样, 当 $ {z_2} \ne 0 $ 时, 
\begin{align*}
\frac{{{z_1}}}{{{z_2}}} = \frac{{{z_1}{{\bar z}_2}}}{{{z_2}{{\bar z}_2}}} &= \frac{1}{{{{\left| {{z_2}} \right|}^2}}}\left| {{z_1}} \right| \cdot \left| {{z_2}} \right|[\cos ({\theta _1} - {\theta _2}) + {\rm{i}}\sin ({\theta _1} - {\theta _2})]\\
&= \frac{{\left| {{z_1}} \right|}}{{\left| {{z_2}} \right|}}[\cos ({\theta _1} - {\theta _2}) + {\rm{i}}\sin ({\theta _1} - {\theta _2})]
\end{align*}

于是
\[ {\rm{ }}\left| {\frac{{{z_1}}}{{{z_2}}}} \right| = \frac{{\left| {{z_1}} \right|}}{{\left| {{z_2}} \right|}}, {\rm{Arg}}\left( {\frac{{{z_1}}}{{{z_2}}}} \right) = {\rm{Arg}}{z_1} - {\rm{Arg}}{z_2}.\]
\end{frame}

\begin{frame}[t]
\frametitle{复数的代数运算}
\textbf{乘幂与方根}\\
两个复数商的模等于它们模的商; 两个复数商的辐角等于被除数与除数的辐角之差.
设复数 $ {z_1} $ 和 $ {z_2} $ 的指数形式分别为
\[ {z_1} = {r_1}{e^{i{\theta _1}}}, {z_2} = {r_2}{e^{i{\theta _2}}}, \]
则
\[\frac{{{z_1}}}{{{z_2}}} = \frac{{{r_1}}}{{{r_2}}}{e^{i({\theta _1} - {\theta _2})}}.\]
\textbf{除法的几何意义}\\

先讨论 $ 1/z $ 的几何意义. 将 $ 1/z $ 分解为如下两个运算
\[ {{z'}} = 1/{{\bar z}} \qquad {{w}} = {{\bar z'}}\]

\begin{minipage}{0.55\linewidth}
如果 $ {{z}} = {{r}}{e^{{{i\theta }}}} $ 
则 $ {{z'}} = \frac{1}{{{r}}}{e^{{{i\theta }}}} $ 
即 $ \arg z'=\arg z, |zz'|=1 $ 
如果 $ |z|<1 $,  $ z $ 在单位圆内, $ z' $ 在单位圆外, 且与 $ z $ 位于同一条由原点发出的射线上
\end{minipage}\qquad
\begin{minipage}{0.25\linewidth}
\includegraphics[width=\linewidth]{44}
\end{minipage}

\end{frame}




\begin{frame}[t]
\frametitle{复数的代数运算}
\textbf{$ z' $ 的求法}\\

过 $ z $ 点做 $ oz $ 的垂线, 与单位圆相交于 $ T $. \\
过 $ T $ 做单位圆周的切线, 与射线 $ oz $ 相交点即为 $ z' $. 

\medskip
\medskip

\begin{minipage}[b]{0.5\linewidth}

\textbf{ $ 1/z $ 的求法}
做 $ z' $ 的共轭复数即可得到 $ 1/z $. 

 $ |z|>1 $ 的做法自己完成. 

 $ |z|=1 $ 时, $ z=z' $, 两点重合. 
\end{minipage}\qquad
\begin{minipage}[b]{0.3\linewidth}
\includegraphics[width=\linewidth]{45}
\end{minipage}

由 $ z $ 点得到 $ z' $ 点, 称为关于单位圆的对称映射, 或对称变换. 

 $ z' $ 的共轭复数是 $ z' $ 关于实轴的对称映射. 

\alert{因此, 从映射角度讲, $ w =1/z $ 是关于单位圆周的对称映射和关于实轴的对称映射的复合映射. 

得到 $ w =1/z $ 的几何意义后, 很容易给出除法的几何意义. }
\end{frame}



\begin{frame}[t]
\frametitle{复数的代数运算}
\textbf{乘幂与方根}\\
\begin{block}{ $ n $ 次幂}
 $ n $ 个相同复数 $ z $ 的乘积称为 $ z $ 的 $ n $ 次幂, 记作 $ {z^n} $. 

\[{z^n} = \underbrace {z \cdot z \cdot \cdots \cdot z}_{n~\text{个}}.\]

对任一正整数 $ n $,  有:  $ {z^n} = {r^n}(\cos n\theta + i\sin n\theta ). $ 

若定义 $ {z^{ - n}} = \frac{1}{{{z^n}}} $,  则当 $ n $ 为负整数时,  上式仍成立.

特别, 当 $ \left| z \right| = 1 $ 时, 即 $ z = \cos \theta + i\sin \theta,  $ 

 $ {(\cos \theta + {\rm{i}}\sin \theta )^n} = \cos n\theta + {\rm{i}}\sin n\theta $ de Moivr 公式

\end{block}
\end{frame}

\begin{frame}[t]
\frametitle{复数的代数运算}

\begin{exampleblock}{例5: 化简 $ (1+i)^n + (1-i)^n$.}
解:
$$(1+i) = \sqrt{2}\left( \frac{\sqrt{2}}{2} + \frac{\sqrt{2}}{2} i \right) =  \sqrt{2} \left[ \cos{\frac{\pi}{4}}+ \sin{\frac{\pi}{4}} \right]$$
$$(1-i) = \sqrt{2}\left( \frac{\sqrt{2}}{2} - \frac{\sqrt{2}}{2} i \right) =  \sqrt{2} \left[ \cos{\left(-\frac{\pi}{4} \right)}+ \sin{\left(-\frac{\pi}{4} \right)} \right]$$
\begin{eqnarray*} 
%\begin{align*}
 & & (1  + i)^n + (1-i)^n  \\
 &=& (\sqrt{2})^n \left[ \cos{\frac{\pi}{4}}+ \sin{\frac{\pi}{4}}\right]^n+(\sqrt{2})^n \left[ \cos{\left( -\frac{\pi}{4}  \right)} + \sin{\left(-\frac{\pi}{4} \right)} \right]^n \\
 &=& (\sqrt{2})^n \left[\cos{\frac{n\pi}{4}}+i\sin{\frac{n\pi}{4}} +\cos{\frac{n\pi}{4}}-i\sin{\frac{n\pi}{4}}  \right]\\
 &=&  2^{\frac{n+2}{2}}\cos{\frac{n\pi}{4}} 
\end{eqnarray*}
%\end{align*}
\end{exampleblock}
\end{frame}


\begin{frame}[t]
\frametitle{复数的代数运算}
\textbf{乘幂与方根}\\
\begin{block}{ $ n $ 次方根}
给定复数 $ z $, 方程 $ {w^n} = z $ 的根称为 $ z $ 的 $ n $ 次方根, 

记为 $ \sqrt[n]{z} $. 可以推得: 
\[{w_k} = \sqrt[n]{z} = {r^{\frac{1}{n}}}\left( {\cos \frac{{\theta + 2k\pi}}{n} + i\sin \frac{{\theta + 2k\pi}}{n}} \right)~~~(\; k = 0, \; 1, \; 2, \; \cdots, \; n - 1\; )\]

从几何上看,  $ \sqrt[n]{z} $ 的 $ n $ 个值就是以原点为中心,  $ {r^{\frac{1}{n}}} $ 为半径的圆的内接正 $ n $ 边形的 $ n $ 个顶点. 其中一个顶点的幅角为 $  \theta/n $ 

\end{block}
\end{frame}

\begin{frame}[t]
\frametitle{复数的代数运算}
\textbf{乘幂与方根}\\
推导过程如下: 

设 
\begin{gather*}
z = r(\cos \theta + i\sin \theta ), \\
w = \rho (\cos \phi + i\sin \phi ), \\
{\rho ^n}(\cos n\phi + i\sin n\phi ) = r(\cos \theta + i\sin \theta ) 
\end{gather*}

于是 $ {\rho ^n} = r, \cos n\phi = \cos \theta,  \sin n\phi = \sin \theta,  $ 显然 $ n\phi = \theta + 2k\pi $,  $ (k = 0, \; \pm 1, \; \pm 2, \cdots ) $ 

故 $ \rho = {r^{\frac{1}{n}}}, \quad \phi = \frac{{\theta + 2k\pi}}{n}, $ 

\[w = \sqrt[n]{z} = {r^{\frac{1}{n}}}\left( {\cos \frac{{\theta + 2k\pi}}{n} + i\sin \frac{{\theta + 2k\pi}}{n}} \right)~~~~ (k = 0, \; \pm 1, \; \pm 2, \cdots )\]
\end{frame}


\begin{frame}[t]
\frametitle{复数的代数运算}
\textbf{乘幂与方根}\\
当 $ k = 0, 1, 2, \cdots, n - 1 $ 时,  得到 $ n $ 个相异的根: 
\begin{align*}
&{w_0} = {r^{\frac{1}{n}}}\left( {\cos \frac{\theta }{n} + i\sin \frac{\theta }{n}} \right), \\
&{w_1} = {r^{\frac{1}{n}}}\left( {\cos \frac{{\theta + 2\pi}}{n} + i\sin \frac{{\theta + 2\pi}}{n}} \right), \\
&{w_{n - 1}} = {r^{\frac{1}{n}}}\left( {\cos \frac{{\theta + 2(n - 1)\pi}}{n} + i\sin \frac{{\theta + 2(n - 1)\pi}}{n}} \right).
\end{align*}

当 $ k $ 以其他整数值代入时, 这些根又重复出现. 
\[ {w_n} = {r^{\frac{1}{n}}}\left( {\cos \frac{{\theta + 2n\pi}}{n} + i\sin \frac{{\theta + 2n\pi}}{n}} \right) = {r^{\frac{1}{n}}}\left( {\cos \frac{\theta }{n} + i\sin \frac{\theta }{n}} \right)\]
\end{frame}

\begin{frame}[t]
\frametitle{复数的代数运算}
\begin{exampleblock}{例:计算 $ \sqrt[{{4}}]{{1 + i}} $ 的值.}

解:  $ 1 + i = \sqrt 2 \left[ {\cos \frac{\pi }{4} + i\sin \frac{\pi }{4}} \right] $. 
$$\sqrt[4]{{1 + i}} = \sqrt[8]{2}\left[ {\cos \frac{{\frac{\pi }{4} + 2k\pi }}{4} + i\sin \frac{{\frac{\pi }{4} + 2k\pi }}{4}} \right]~~~~
(k = 0, \; 1, \; 2, \; 3)$$
即\begin{minipage}{0.55\linewidth}

\begin{align*}
&{w_0} = \sqrt[8]{2}\left[ {\cos \frac{\pi }{{16}} + i\sin \frac{\pi }{{16}}} \right], \\
&{w_1} = \sqrt[8]{2}\left[ {\cos \frac{{9\pi }}{{16}} + i\sin \frac{{9\pi }}{{16}}} \right], \\
&{w_2} = \sqrt[8]{2}\left[ {\cos \frac{{17\pi }}{{16}} + i\sin \frac{{17\pi }}{{16}}} \right], \\
&{w_3} = \sqrt[8]{2}\left[ {\cos \frac{{25\pi }}{{16}} + i\sin \frac{{25\pi }}{{16}}} \right].
\end{align*}
\end{minipage}\qquad
\begin{minipage}{0.35\linewidth}
\includegraphics[width=\linewidth]{52}
这四个根是内接于中心在原点半径为 $ \sqrt[{8}]{{2}} $ 的圆的正方形的四个顶点.
\end{minipage}
\end{exampleblock}

\end{frame}


\begin{frame}[t]
\frametitle{复数的代数运算}
\begin{exampleblock}{例: 解方程 $(1+z)^5 = (1-z)^5 $}
解:直接可验证方程的根$z \neq 1$,故原方程可转换为:
$ \left( \frac{1+z}{1-z}\right)^5 = 1 $\\
另$w = \frac{1+z}{1-z}$,\\
则$w^5 = 1, w = e^{\frac{2k\pi}{5}}, k = 0, 1, 2, 3, 4$.\\
$w_0 = 1, w_1 = e^{\frac{2\pi}{5}}, w_2 = e^{\frac{4\pi}{5}}, w_1 = e^{\frac{6\pi}{5}},w_1 = e^{\frac{8\pi}{5}}$\\
因为
\begin{eqnarray*}
z = \frac{w-1}{w+1} & = & \frac{e^{i \alpha}-1}{e^{i \alpha}+1} = \frac{\cos{\alpha}+i\sin{\alpha}-1}{cos{\alpha}+i\sin{\alpha}+1}\\
& = & \frac{\sin{\frac{\alpha}{2}}(-\sin{\frac{\alpha}{2}}+i\cos{\frac{\alpha}{2}})}{\cos{\frac{\alpha}{2}}(\sin{\frac{\alpha}{2}}+i\cos{\frac{\alpha}{2}})}
\end{eqnarray*}
故原方程的根为:$z_0 = 0, z_1 = i \tan{\frac{\pi}{5}}, z_2 = i \tan{\frac{2\pi}{5}}, z_3 = i \tan{\frac{3\pi}{5}}, z_4 = i \tan{\frac{4\pi}{5}}$
\end{exampleblock}
\end{frame}

\begin{frame}{复数的代数运算}
\begin{alertblock}{复数域}
全体复数并引进了上述加减乘除和取共轭运算后称为复数域, 记为 $ \mathbb{C} $.     

实数域是复数域的一部分(子域). 

在复数域, 方程 $ x^2 = - 1 $ 有解, 而在实数域无解. 

在复数域, 两个复数不能比较大小. 即: 有得、有失.     

域: 对加法和乘法封闭. 

\end{alertblock}
\end{frame}



\subsubsection{复球面与无穷远点}

\begin{frame}[t]
\frametitle{课程目录}
\tableofcontents[currentsubsection ] % Throughout your presentation, if you choose to use \section{} and \subsection{} commands, these will automatically be printed on this slide as an overview of your presentation
\end{frame}

\begin{frame}{复球面与无穷远点}
\begin{minipage}{0.5\linewidth}
取一个与复平面切于原点 $ z =0 $ 的球面球面上一点 $ S $ 与原点重合 (如图),  通过 $ S $ 作垂直于复平面的直线与球面相交于另一点 $ N $,  称 $ N $ 为北极,  $ S $ 为南极.
\end{minipage}\qquad\qquad
\begin{minipage}{0.3\linewidth}
\includegraphics[width=\linewidth]{53}
\end{minipage}
\end{frame}

\begin{frame}{复球面与无穷远点}
\begin{figure}
\includegraphics[width=0.85\linewidth]{54}
\end{figure}
球面上的点,  除去北极 $ N $ 外,  与复平面内的点之间存在着一一对应的关系. 我们用球面上的点来表示复数.

\bigskip

球面上的北极 $ N $ 不能对应复平面上的定点, 但球面上的点离北极 $ N $ 越近, 它所表示的复数的模越大.

\end{frame}

\begin{frame}{复球面与无穷远点}
\begin{figure}
    \includegraphics[width=0.3\linewidth]{55}
\end{figure}

我们规定: 复数中有一个唯一的``无穷大''与复平面上的无穷远点相对应, 记作 $ \infty $. 

\bigskip

因而,   球面上的北极 $ N $ 就是复数无穷大的几何表示.

\end{frame}

\begin{frame}{复球面与无穷远点}
\begin{figure}
    \includegraphics[width=0.3\linewidth]{55}
\end{figure}
\begin{alertblock}{复平面:不包括无穷远点的复平面称为有限复平面, 或简称复平面.}
\end{alertblock}
\bigskip
\begin{alertblock}{扩充复平面:包括无穷远点的复平面称为扩充复平面.}


\bigskip

球面上的每一个点与扩充复平面的每一个点构成了一一对应, 这样的球面称为复球面.

\bigskip

引入复球面后, 能将扩充复平面的无穷远点明显地表示出来.
\end{alertblock}

\end{frame}

\begin{frame}{复球面与无穷远点}
对于复数的无穷远点而言, 它的实部, 虚部,  辐角等概念均无意义, 规定它的模为正无穷大.

关于 $ \infty $ 的四则运算规定如下: 

\begin{enumerate}[(1)]
    \item 加法:  $ \alpha + \infty = \infty + \alpha = \infty, \quad (\alpha \ne \infty ) $ 
    \item 减法:  $ \alpha - \infty = \infty - \alpha = \infty, \quad (\alpha \ne \infty ) $ 
    \item 乘法:  $ \alpha \cdot \infty = \infty \cdot \alpha = \infty, \quad (\alpha \ne 0) $ 
    \item 除法:  $ \frac{\alpha }{\infty } = 0, \quad \frac{\infty }{\alpha } = \infty, \; (\alpha \ne \infty ), \quad \frac{\alpha }{0} = \infty, (\alpha \ne 0) $ 
\end{enumerate}

引入无穷远点 $ \infty $ 后, 许多我们习以为常的概念需要修订. 

如: 所有直线在∞相交. 

如: 一动点沿直线运动后会回到出发点. 

\end{frame}

\subsection{复平面上的曲线和区域}
\begin{frame}[t]
\frametitle{课程目录}
\tableofcontents[currentsubsection ] % Throughout your presentation, if you choose to use \section{} and \subsection{} commands, these will automatically be printed on this slide as an overview of your presentation
\end{frame}

\subsubsection{曲线的复数方程}
\begin{frame}[t]
\frametitle{课程目录}
\tableofcontents[currentsubsection ] % Throughout your presentation, if you choose to use \section{} and \subsection{} commands, these will automatically be printed on this slide as an overview of your presentation
\end{frame}

\begin{frame}{曲线的复数方程}
\textbf{复平面上曲线方程有两种表示方式}
\begin{itemize}
    \item 直角坐标方程
    \item 参数方程
\end{itemize}

\end{frame}

\begin{frame}{曲线的复数方程}
\textbf{复平面上曲线$C$直角坐标方程的复数形式}\\
设平面上曲线$C$的方程为$F(x,y) = 0$\\
由
$$x = \frac{z + \bar{z}}{2}, y= \frac{z - \bar{z}}{2}$$
得
$$F\left( \frac{z + \bar{z}}{2},  \frac{z - \bar{z}}{2} \right) = 0$$
或
$$F\left( \mathbf{Re}(z), \mathbf{Im}(z) \right )= 0$$


\end{frame}

\begin{frame}{曲线的复数方程}

\begin{exampleblock}{例: 试用复数表示圆的方程 \\$A(x^2+y^2) +Bx + Cy +D = 0$, 其中$A, B, C, D$是常数.}

解: 由 $x = \frac{z+\bar{z}}{2}, y = \frac{z-\bar{z}}{2i}, x^2+y^2 = |z|^2=z\bar{z}$\\
得 $Az\bar{z}+B\frac{z+\bar{z}}{2}+C\frac{z-\bar{z}}{2i}+D = 0 $\\
令$\beta = \frac{B+Ci}{2}$,上式可化简为$Az\bar{z}+\bar{\beta} z + \beta \bar{z} +D =0$\\
其中$A,D$为实常数,$\beta$为复常数且$|\beta|^2>AD$, $A$不为零.\\
如果$A=0$,$B,C$不全为零,上述方程变为直线方程:
$$\bar{\beta} z + \beta \bar{z} +D =0$$
即为复平面上直线方程的一板形式。
\end{exampleblock}
 
\end{frame}

\begin{frame}{曲线的复数方程}

\begin{enumerate}
	\item 用复数的实部或虚部的等式表示曲线\\
	$\mathbf{Re}(z-z_0)=a $是$XOY$平面上的直线: $x=a+\mathbf{Re}(z_0)$\\
	$\mathbf{Im}(z-z0)=b$是$XOY$平面上的直线: $y=b+\mathbf{Im}(z_0)$\\
	如:$Im(z-i2)=2$
\begin{figure}
\includegraphics[width=0.3\linewidth]{Curve01.jpeg}
\end{figure}
\end{enumerate}
\end{frame}

\begin{frame}{曲线的复数方程}

\begin{enumerate}
	\setcounter{enumi}{1}
	\item 用复数模的等式表示曲线\\


\begin{example}
过两点 $ z_1 $ 和 $ z_2 $ 的线段的垂直平分线方程
\[ |z - z_1 | = | z - z_2| \]	
\end{example}
 
\begin{example}
以原点为圆心, $ R $ 为半径的圆周的方程
\[ |z| = R \]
\end{example}

\begin{example}
以 $  z_0 \neq 0 $ 为圆心, $ R $ 为半径的圆周的方程
\[ |z - z0| = R \]	
\end{example}
\begin{figure}
\includegraphics[width=0.3\linewidth]{Curve01.jpeg}
\end{figure}
\end{enumerate}
\end{frame}


\begin{frame}{曲线的复数方程}



\begin{example}

\begin{columns}
\column{0.5\textwidth}
焦点是 $ z_1 $ 和 $ z_2 $ 的椭圆
\[ |z - z_1 | + | z - z_2| = 2R \]
$R$为椭圆的半长轴.

\column{0.5\textwidth}
\begin{figure}[htbp]
\includegraphics[width=0.5\linewidth]{Curve02}
%\caption{Graph of $f(x)=ax^2|_{\{0.1, 0.3, 1.0, 3.0\}}$}
\end{figure}
\end{columns}


\end{example}


\end{frame}


\begin{frame}{曲线的复数方程}
\begin{example}
焦点是 $ z_1 $ 和 $ z_2 $ 的双曲线: 
\[ |z - z_1 | - | z - z_2| = 2R \]
$R$为双曲线的实半轴.
\end{example}

\begin{example}
准线为 $ x=c $, 对称轴为实轴的抛物线方程为
\begin{figure}
\includegraphics[width=0.3\linewidth]{37}
\end{figure}
\end{example}
\end{frame}


\begin{frame}{曲线的复数方程}

\begin{enumerate}
	\setcounter{enumi}{2}
	\item 用复数辐角的等式表示曲线\\
	从点$z_0$出发,与实轴夹角$\theta_0$的射线为:
	$$arg (z-z_0) = \theta_0,\quad (-\pi < \theta_0 \leq \pi)$$
	看两个例子:\\
	$$arg (z-i) = \frac{\pi}{4} \quad  \quad  \quad \frac{\pi}{4} < arg (z) < \frac{3\pi}{4} $$
\begin{figure}
\includegraphics[width=0.6 \textwidth]{Curve03}
\end{figure}
\end{enumerate}
\end{frame}

\begin{frame}{曲线的复数方程}
\textbf{复平面上曲线$C$的参数方程为}\\
$$C: x=x(t), y=y(t) (\alpha \leq t \leq \beta)$$
那么,复平面上曲线$C$的动点$z(t)$可表示为:
$$C: z= x (t) + iy(t) $$
或
$$z=z(t) (\alpha \leq t \leq \beta)$$
依赖于参数$t$.

\begin{example}
通过复平面上两 $ z_1 $,  $ z_2 $ 点的直线方程
\[ z = z_1+ t (z_2-z_1)\qquad -\infty<t <\infty \]
\end{example}

\end{frame}

\subsubsection{复平面上的点集}
\begin{frame}
\frametitle{课程目录}
\tableofcontents[currentsubsection ] % Throughout your presentation, if you choose to use \section{} and \subsection{} commands, these will automatically be printed on this slide as an overview of your presentation
\end{frame}

\begin{frame}{复平面上的点集}
设有复平面上一些点的集合, 称为复数集或点集$E$. 
如果 $ z $ 是点集 $ E $ 中的一个点, 记为 $ z\in E $. 

\bigskip

如果点集 $ F $ 中的每一个点都是点集 $ E $ 中的点, 则称 $ E $ 包含 $ F $,  记为 $ E \supset F $. 或称 $ F $ 包含于 $ E $, 记为 $ F \subset E  $ 

\bigskip

$ E $ 包含 $ F $ 的数学描述为
\[ E \supset F \Leftrightarrow \forall\: z\in F \Rightarrow z\in E \]

\end{frame}

\begin{frame}{复平面上的点集}
\begin{alertblock}{(1) 邻域}
复平面上以 $ z_0 $ 为中心, $ \delta > 0 $ 为半径的圆的内部 $ |z - z_0| < \delta $ 的点的集合称为 $ z_0 $ 的邻域, 记作 $ N(z_0, \delta) $. 

称 $ 0<|z - z_0|<\delta $ 的点的集合称为 $ z_0 $ 的去心邻域, 记作 $ N_0(z_0, \delta) $. 
\end{alertblock}

\begin{block}{注意}
设 $ R>0 $, 满足 $ |z|>R $ 的所有点的集合(包括无穷远点本身)称为无穷远点的邻域. 

设 $ R>0 $, 满足 $ |z|>R $ 的点的集合(不包括无穷远点)称为无穷远点的去心邻域. 可以表示为    
\[ R<|z|<+\infty. \]

\end{block}

\end{frame}

\begin{frame}{复平面上的点集}
\begin{alertblock}{(2) 聚点}
如果 $ z $ 的任意小的邻域中, 恒有 $ E $ 中无穷多的点, 称 $ z $ 是 $ E $ 的聚点. 

在 $ z $ 的任意小的邻域中, 恒有一个不等于 $ z $ 的、属于 $ E $ 的点, 称 $ z $ 是 $ E $ 的聚点. 
例: $ E={1/n}, n=1, 2, \dots, z=0 $ 是E的聚点, 但 $ z\notin E $ 

\end{alertblock}

\begin{alertblock}{(3) 孤立点}
如果 $ z\in E $, 而 $ z $ 不是 $ E $ 的聚点, 称 $ z $ 是 $ E $ 的孤立点. 

\end{alertblock}

\begin{alertblock}{(4) 内点}
如果 $ z $ 有某邻域 $ F $, 使得 $ F\in E $, 称 $ z $ 是 $ E $ 的内点. 
\end{alertblock}

\end{frame}

\begin{frame}{复平面上的点集}
\begin{alertblock}{(5) 边界点}
	如果在 $ z $ 的任意小的邻域中, 既有 $ E $ 中的点, 又有非 $ E $ 中的点, 称 $ z $ 是 $ E $ 的边界点. 
	
	例: $ E={1/n| n=1, 2, \dots} $,  $ E $ 中所有的点都是 $ E $ 的边界点, 另外 $ z=0 $ 也是 $ E $ 的边界点. 
	
	孤立点是边界点. 
	
	不属于 $ E $ 的聚点也是边界点. 
\end{alertblock}

\begin{alertblock}{(6) 开集}
全由内点组成的点集称为开集. 

\end{alertblock}


\end{frame}

\begin{frame}{复平面上的点集}
\begin{alertblock}{(7) 闭集}
	如果 $ E $ 的所有聚点都属于 $ E $, 称 $ s $ 为闭集. 
	
	例: 邻域是开集. 
	
	例: $ E=0, 1/n, n=1, 2, \dots  $,  是闭集
	
	例: $ E={1/n}, n=1, 2, \dots  $. 既不是开集, 也不是闭集. 
\end{alertblock}

\begin{alertblock}{(8) 有界集}
	对 $ E $ 中所有的点 $ z $, 如果存在 $ M>0 $, 使得 $ |z|<M $, 称 $ E $ 为有界集. 
	
	两种不同的数学描述
	
	 $ \exists\: M>0 $, 对 $ \forall\: z \in E $, 有 $ |z|<M $, 称 $ E $ 为有界集. 
	
	对 $ \forall\: z\in 0 $,  $ \exists\: M>0  $, 有 $ |z|<M $, 称 $ E $ 为有界集. 
	
\end{alertblock}

\begin{alertblock}{(9) 连通集}
	如果 $ E $ 中任意两点之间可用一条完全包含在E中的连续线相连, 称$E$为连通集. 
	
\end{alertblock}

\end{frame}

\subsubsection{区域和曲线}

\begin{frame}
\frametitle{课程目录}
\tableofcontents[currentsubsection ] % Throughout your presentation, if you choose to use \section{} and \subsection{} commands, these will automatically be printed on this slide as an overview of your presentation
\end{frame}

\begin{frame}{区域和曲线}
\begin{alertblock}{(1) 区域}
连通的开集称为区域. 
\end{alertblock}

\begin{alertblock}{(2) 区域的边界, 闭区域}
\begin{minipage}{0.6\linewidth}
区域 $ D $ 的全体边界点的集合 $ C $ 组成 $ D $ 的边界.  

区域 $ D $ 与它的边界 $ C $ 一起构成闭区域 $ \bar{D} $. 
\[{{\bar D}} = {{D}} + {{C}}\]
\end{minipage}\qquad
\begin{minipage}{0.3\linewidth}
\begin{figure}
	\includegraphics[width=\linewidth]{67}
\end{figure}
\end{minipage}
\end{alertblock}

\end{frame}

\begin{frame}{区域和曲线}
以上基本概念的图示
\begin{figure}
	\includegraphics[width=0.75\linewidth]{68}
\end{figure}
\end{frame}

\begin{frame}{区域和曲线}
\begin{alertblock}{(3) 区域边界的正向}
如果沿区域 $ D $ 的边界 $ C $ 行进, 区域 $ D $ 一直在左方, 则此行进方向称为边界 $ C $ 的正向

例: 圆内域的边界正向是逆时针方向. 

例: 圆外域的边界正向是顺时针方向. 

\end{alertblock}

\begin{alertblock}{(4) 平面曲线}
如果 $ x(t) $ 和 $ y(t) $ 是两个连续的实函数, 那末方程组 $ x = x(t) $,   $ y = y(t) $,  $ (a \le t \le b) $ 代表一条平面曲线, 称为连续曲线.

平面曲线的复数表示: 
\[z = z(t) = x(t) + {\rm{i}}y(t).\quad (a \le t \le b)\]
\end{alertblock}
\end{frame}

\begin{frame}{区域和曲线}
\begin{alertblock}{(5) 可求长曲线}
设曲线 $ z = z(t) (a<t<b ) $ 在 $ [a, b] $ 上连续, 如果任取实数序列 $ \{t_n\} $ 
\[{{a}} = {{{t}}_0} < {{{t}}_1} < \cdots < {{{t}}_{{n}}} = {{b}}\]

其在曲线上对应的点列为 $ {{{z}}_{{k}}} = {{z}}\left( {{{{t}}_{{k}}}} \right)~~{{k}} = 0, 1, \cdots, {\rm{n}} $ 

将这些点用折线 Q 依次连接起来, Q的长度
\[{{L}} = \sum\limits_{{{k}} = 1}^{{n}} {|{{z}}({{{t}}_{{k}}}) - {{z}}({{{t}}_{{{k}} - 1}})|} \]

如果对任意长度的 $ n $ 的实序列 $ \{t_n\} $,  $ L $ 有上界, 称曲线 $ z = z(t) $ 为可求长曲线. 

 $ L $ 的上确界即为 $ z(t) $ 的长度. 

\end{alertblock}
\end{frame}

\begin{frame}{区域和曲线}
\begin{alertblock}{(6) 实变量复值函数的导数}
如果实变量复值函数  $ z(t) = x(t) + iy(t) $ 的实部 $ x(t) $ 和虚部 $ y(t) $ 均可导, 称
\[z'(t) = x'(t) + {\rm{i}}y'(t)\]
为实变量复值函数  $ z(t) $ 的导数. 

实变量复值函数求导为分别对实部和虚部求导. 


\end{alertblock}
\end{frame}

\begin{frame}{区域和曲线}
\begin{alertblock}{(7) 按段 光滑曲线}
如果在 $ [a, b] $ 上, $ x'(t) $ 和 $ y'(t) $ 都是 $ t $ 的连续函数, 即, $ z'(t)=x'(t)+iy'(t) $ 是 $ t $ 的连续函数, 且对每一个t值, 有 $ [x'(t)]^2+[y'(t)]^2 \neq 0 $, 即 $ |z'(t)| \neq 0 $, 称此曲线为光滑曲线. 

\bigskip

由几段依次相接的光滑曲线所组成的曲线称为按(分)段光滑曲线.
\begin{figure}
\includegraphics[width=0.8\linewidth]{73}
\end{figure}
\end{alertblock}
\end{frame}

\begin{frame}{区域和曲线}
\begin{alertblock}{(8) 简单曲线}
对连续曲线 $ C: z=z(t)~~ t\in [a, b], z(a), z(b) $ 分别称为 $ C $ 的起点和终点. 

当 $ t1\neq t2 $ 而有 $ z(t_1)=z(t_2) $ 时, 点 $ z(t_1) $ 称为曲线 $ C $ 的重点. 

没有重点的连续曲线 $ C $ 称为简单曲线.

除 $ z(a)= z(b) $ 外别无重点的连续曲线 称为简单闭曲线.

简单闭曲线也称为 Jordan (若当) 曲线.

\end{alertblock}

\begin{theorem}{(9) Jordan定理}
任意一条简单闭曲线 $ C $ 将复平面唯一地分成 $ C $,  $ I(C) $,  $ E(C) $ 三个互不相交的点集. 
\begin{figure}\vspace{-5mm}
\includegraphics[width=0.3\linewidth]{75}
\end{figure}
\end{theorem}
\end{frame}

\begin{frame}{区域和曲线}
\begin{block}{它们有如下性质: }
\begin{enumerate}
	\item 彼此不交. 
	\item $ I(C) $ 是一个有界区域, 称为 $ C $ 的内部. 
	\item  $ E(C) $ 是一个无界区域, 称为 $ C $ 的外部. 
	\item $ C $ 是 $ I(C) $ 的边界, 也是 $ E(C) $ 的边界. 
	\item 如果简单折线 $ p $ 的一个端点属于 $ I(C) $, 另一个端点属于 $ E(C) $, 则 $ p $ 与 $ C $ 必有交点.
\end{enumerate}

\end{block}
\end{frame}

\begin{frame}{区域和曲线}
\begin{block}{课堂练习: 判断下列曲线是否为简单曲线?}
\begin{figure}
\includegraphics[width=0.85\linewidth]{77}
\end{figure}
\end{block}
\end{frame}

\begin{frame}{区域和曲线}
\begin{alertblock}{(10) 单连通域与多连通域的定义}
复平面上的一个区域 $ D $,   如果在其中任作一条简单闭曲线,  而曲线的内部总属于 $ D $,   就称为单连通区域. 一个区域如果不是单连通域, 就称为多连通区域. 
\begin{figure}\centering
\includegraphics[width=0.85\linewidth]{78}\\
\qquad\quad 单连通域 \qquad\qquad\qquad\qquad\qquad\quad 多连通域
\end{figure}
\end{alertblock}
\end{frame}

\begin{frame}{区域和曲线}
\begin{alertblock}{$ n( n >2) $ 维空间中的单连域}
如果区域 $ D $ 任一条简单闭曲线都可以在 $ D $ 的内部连续收缩成 $ D $ 的一个点, 称 $ D $ 为单连通区域. 

\begin{example}
	复平面上的圆外域. 
\end{example}

\begin{example}
	三维空间的球外域. 	
\end{example}

\begin{example}
	三维空间的车胎域. 	
\end{example}
\end{alertblock}

\end{frame}

\begin{frame}{区域和曲线}
\begin{block}{例 1}
指出下列不等式所确定的点集, 是有界的还是无界的, 单连通的还是多连通的.
\begin{align*}
& (1)~{\rm Re} ({z^2}) < 1; \quad (2)~\left| {\arg z} \right| < \frac{\pi }{3}; \quad (3)~\left| {\frac{1}{z}} \right| < 3; \\
&(4)~\left| {z - 1} \right| + \left| {z + 1} \right| < 4; \quad \quad (5)~\left| {z - 1} \right| \cdot \left| {z + 1} \right| < 1.
\end{align*}
\bigskip

解 (1) \begin{minipage}[t]{0.5\linewidth}
当 $ z = x + iy $ 时, 
 $ {\rm Re} ({z^2}) = {x^2} - {y^2}, {\rm Re} ({z^2}) < 1 \Leftrightarrow {x^2} - {y^2} < 1 $,  无界的单连通域(如图).
\end{minipage}\qquad
\begin{minipage}[t]{0.3\linewidth}
\begin{figure}
\includegraphics[width=\linewidth]{80}
\end{figure}
\end{minipage}

\end{block}
\end{frame}

\begin{frame}{区域和曲线}
\begin{block}{}
\begin{minipage}[t]{0.5\linewidth}
(2) $ \left| {\arg z} \right| < \frac{\pi }{3} $ 
 $ \left| {\arg z} \right| < \frac{\pi }{3} \Leftrightarrow - \frac{\pi }{3} < \arg z < \frac{\pi }{3}, $ 是角形域, 无界的单连通域(如图).
\vspace{25mm}

(3) $ \left| {\; \frac{1}{z}\; } \right| < 3 ~~~~ \left| \frac{1}{z} \right| < 3 \Leftrightarrow \left| z \right| > \frac{1}{3}, $ 
是以原点为中心, 半径为 $ \frac{{1}}{{3}} $ 的圆的外部, 
\end{minipage}\qquad\qquad
\begin{minipage}[t]{0.15\linewidth}
\begin{figure}\vspace{-3mm}\centering
\includegraphics[width=\linewidth]{81a}\\
\includegraphics[width=1.5\linewidth]{81b}\\
\end{figure}
\end{minipage}

\end{block}

\end{frame}

\begin{frame}{区域和曲线}
\begin{block}{}
(4) $ \left| {z - 1} \right| + \left| {z + 1} \right| < 4 $ 

因 $ \left| {z - 1} \right| + \left| {z + 1} \right| = 4 $ 

表示到1,  $ -1 $ 的距离之和为定值 4 的点的轨迹, 是椭圆, 

 $ \left| {z - 1} \right| + \left| {z + 1} \right| < 4 $ 表示该椭圆内部, 有界的单连通域.

\begin{figure}
	\includegraphics[width=0.3\linewidth]{82}
\end{figure}
\end{block}


\end{frame}

\begin{frame}{区域和曲线}
\begin{block}{}
(5) $ \left| {z - 1} \right| \cdot \left| {z + 1} \right| < 1 $ 

令 $ z = r\cos \theta + ir\sin \theta $, 
\begin{align*}
&\left| {z - 1} \right| \cdot \left| {z + 1} \right| < 1 \Leftrightarrow\\
&[{(r\cos \theta - 1)^2} + {r^2}{\sin ^2}\theta ] \cdot [{(r\cos \theta + 1)^2} + {r^2}{\sin ^2}\theta ] < 1\\
&({r^2} + 2r\cos \theta + 1)({r^2} - 2r\cos \theta + 1) < 1\\
&{({r^2} + 1)^2} - 4{(r\cos \theta )^2} < 1 \Rightarrow {r^2} < 2\cos 2\theta, 
\end{align*}
 $ {r^2} = 2\cos 2\theta $ 是双叶玫瑰线 (也称双纽线),  $ \left| {z - 1} \right| \cdot \left| {z + 1} \right| < 1 $ 是其内部, 有界集.但不是区域.
\begin{figure}
\includegraphics[width=0.3\linewidth]{83}
\end{figure}

\end{block}
\end{frame}

\begin{frame}{区域和曲线}
\begin{block}{例 2}
满足下列条件的点集是什么, 如果是区域, 指出是单连通域还是多连通域?

解 \begin{minipage}[t]{0.5\linewidth}
(1) $ {\rm Im} z = 3, $ 是一条平行于实轴的直线, 不是区域.
\vspace{20mm}

(2) $ {\rm Re} z < - 2 $,  
以 $ {\rm Re} z = - 2 $ 为左界的半平面(不包括直线 $ {\rm Re} z = - 2 $), 单连通域.

\end{minipage}\qquad
\begin{minipage}[t]{0.3\linewidth}
\begin{figure}\centering\vspace{-3mm}
\includegraphics[width=0.8\linewidth]{84a}\\
\includegraphics[width=\linewidth]{84b}\\
\end{figure}
\end{minipage}

\end{block}
\end{frame}

\begin{frame}{区域和曲线}
\begin{block}{}
\begin{minipage}[t]{0.5\linewidth}
(3) $ 0 < |z + 1 + i| < 2 $,  以 $ - (1 + i) $ 为圆心, 为半径的去心圆盘, 是多连通域.

\vspace{15mm}

(4) $ \arg (z - i) = \frac{\pi }{4} $,  以 $ i $ 为端点, 斜率为 1 的半射线 (不包括端点 $ i $), 不是区域.

\end{minipage}\qquad
\begin{minipage}[t]{0.25\linewidth}
\begin{figure}\centering\vspace{-3mm}
\includegraphics[width=\linewidth]{85a}\\
\includegraphics[width=\linewidth]{85b}\\
\end{figure}
\end{minipage}

\end{block}
\end{frame}

\subsection{复变函数}
\begin{frame}
\frametitle{课程目录}
\tableofcontents[currentsubsection ] % Throughout your presentation, if you choose to use \section{} and \subsection{} commands, these will automatically be printed on this slide as an overview of your presentation
\end{frame}

\subsubsection{定义与几何意义}

\begin{frame}
\frametitle{课程目录}
\tableofcontents[currentsubsection ] % Throughout your presentation, if you choose to use \section{} and \subsection{} commands, these will automatically be printed on this slide as an overview of your presentation
\end{frame}

\begin{frame}{定义与几何意义}
\begin{block}{定义 1}
设 $ E $ 是一平面点集, 如果存在一个法则, 对任何 $ z = x + iy \in E $,  使得有一个或几个复数 $ w = u + iv $ 与之对应, 那末称复变数 $ w $ 是复变数 $ z $ 的函数(简称复变函数), 记作 $ w = f(z) $. 
\bigskip

如果 $ z $ 的一个值对应着一个 $ w $ 的值, 那末我们称函数 $ f(z) $ 是单值函数.
\bigskip

如果 $ z $ 的一个值对应着两个或两个以上 $ w $ 的值, 那末我们称函数 $ f(z) $ 是多值函数.
\end{block}

\end{frame}
\begin{frame}{定义与几何意义}

例如: $ w = \left| {\, z\, } \right| $ 是复平面上的单值函数, 而函数 $ w = \arg z $ 是定义在 $ z \ne 0 $ 上的多值函数.
\bigskip

复变数 $ w $ 是由两个实变数 $ u, v $ 组成, 同时, 自变量 $ z $ 也是由两个实变数 $ x, y $ 组成, 因此复变函数 $ w $ 与自变量 $ z $ 之间的关系 $ w = f(z) $ 相当于两个关系式: 

\[u = u(x, y), \quad v = v(x, y), \]
\bigskip

它们确定了自变量为 $ x $ 和 $ y $ 的两个二元实变函数.
\end{frame}

\begin{frame}{定义与几何意义}
例如: $ w = {z^2} $ 是一复变函数
令 $ z = x + iy, w = u + iv, {(x + iy)^2}= {x^2} - {y^2} + 2xyi $ 

于是函数 $ w = {z^2} $ 对应于两个二元实变函数: 
\[u = {x^2} - {y^2}, v = 2xy.\]

反过来, 如果 $ w = u(x, y) + iv(x, y) = {x^2} + {y^2} + 4xyi $ 

令 $ x = \frac{{z + \bar z}}{2}, y = \frac{{z - \bar z}}{{2i}} $ 
\begin{align*}
{x^2} + {y^2} + 4xyi &= {\left( {\frac{{z + \bar z}}{2}} \right)^2} + {\left( {\frac{{z - \bar z}}{{2i}}} \right)^2} + 4\frac{{z + \bar z}}{2} \cdot \frac{{z - \bar z}}{{2i}}i\\
& = {z^2} + z \cdot \bar z - {\bar z^2}
\end{align*}

\end{frame}


\begin{frame}{定义与几何意义}
\begin{block}{反函数的定义}
设 $ w = f(z) $ 的定义域为复平面上的集合 $ D $,  函数的值域为复平面上的集合 $ G $,  那末, 对于 $ G $ 中任一 $ w $, 必有 $ D $ 中的一个(或几个)复数与之对应; 这样, 就确定了集合G上的一个单值函数 (或多值函数) $ z = \phi (w),  $ 称它为函数 $ w = f(z) $ 的反函数.
\end{block}
\end{frame}

\subsubsection{极限与连续性}

\begin{frame}
\frametitle{课程目录}
\tableofcontents[currentsubsection ] % Throughout your presentation, if you choose to use \section{} and \subsection{} commands, these will automatically be printed on this slide as an overview of your presentation
\end{frame}


\begin{frame}{极限与连续性}
设复变函数 $ f(z) $ 在集合 $ E $ 上定义, $ z_0 $ 是 $ E $ 的聚点, 当 $ |z - z_0|\to 0 $ 时, 称 $ z \to z_0 $. 

\begin{block}{定义 2}
设复变函数 $ w = f(z) $ 在 $ {z_0} $ 的某个去心邻域 $ 0 < \left| {z - {z_0}} \right| < \rho $ 内定义,  $ A $ 是一个复常数. 若对任意给定的 $ \varepsilon > 0,  $ 总存在 $ \delta (\varepsilon ) > 0~(0 < \delta \le \rho ) $,  使得当 $ 0 < \left| {z - {z_0}} \right| < \delta $ 时, 有 $ \left| {f(z) - A} \right| < \varepsilon $,  那末则称当 $ z $ 趋向于 $ {z_0} $ 时, $ f(z) $ 以 $ A $ 为极限.

记作 $ \lim \limits_{z \to {z_0}} f(z) = A $ 或 $ f(z) \to A~(z \to {z_0}) $ 
\end{block}


注意: 定义中 $ z \to {z_0} $ 的方式是任意的.
\end{frame}
\begin{frame}{极限与连续性}
\begin{block}{例 1}
证明: 当 $ z \to 0 $ 时, 函数 $ f(z) = \frac{z}{{\bar z}} (z \ne 0) $ 的极限不存在.
\end{block}

\begin{block}{证明}
当 $ z $ 沿直线 $ y = kx $ 趋于零时, 
\[\lim \limits_{z \to 0} f(z) = \lim \limits_{x\to 0 \atop 	y= kx} \frac{{x + ikx}}{{x - ikx}} = \frac{{1 + ik}}{{1 - ik}}\]

该极限值随 $ k $ 值的变化而变化, 所以 $ \lim \limits_{z \to 0} f(z) $ 不存在.
\end{block}



\end{frame}

\begin{frame}{极限与连续性}

(1) 连续


\begin{block}{定理1.3.1}
设
\begin{align*}
&f(z) = u(x, y) + iv(x, y), 
&A= u_0+ iv_0, 
&z_0= x_0+ iy_0, 
\end{align*}
则
\begin{gather*}
\lim\limits_{{{z}} \to {{{z}}_{{0}}}} f(z) = A \Leftrightarrow \\
\mathop {{\bf{lim}}}\limits_{\scriptstyle{{x}} \to {{{x}}_{{0}}}\atop	\scriptstyle{{y}} \to {{{y}}_{\bf{0}}}} {{u}} ( {{x}}, {{y}} ) = {{{u}}_0},  \lim\limits_{\scriptstyle{{x}} \to {{{x}}_{{0}}}\atop 	\scriptstyle{{y}} \to {{{y}}_{{0}}}} {{v}} ( {{x}}, {{y}} ) = {{{v}}_0}
\end{gather*}
\end{block}
\end{frame}

\begin{frame}{极限与连续性}
\begin{block}{定理1.3.2}
如果
\[  \lim\limits_{{z} \to {z}_{0}} {f} ( {z} ) = {A} , \lim\limits_{{z} \to {z}_{0}} {g} ( {z} ) = {{B}} \]

则
\begin{enumerate}[1)]
	\item $ \lim\limits_{{z} \to {z}_{0}} [{f} ( {z} ) \pm {g} ( {z})] = {A} \pm {B} $ 
	\item $ \lim\limits_{{z} \to {z}_{0}} [{f} ( {z} ) {g} ( {z})] = {{AB}} $ 
	\item $ \lim\limits_{{z} \to {z}_{0}} [{f} ( {z} ) /{g} ( {z})] = {{A/B}}~~( {{B}} \ne 0 )$ 
\end{enumerate}

\end{block}
\end{frame}

\begin{frame}{极限与连续性}
两个常用符号: 
\bigskip

``O'': 如果 $ \lim \limits_{z \to {z_0}} |f(z)/g(z)| < c(c > 0) $ 

记   $ f(z) = O [g(z)] $ 

\bigskip

``o'': 如果 $ \lim \limits_{z \to {z_0}} [f(z)/g(z)] = 0 $ 

记   $ f(z) = o [g(z)] $ 

\end{frame}

\begin{frame}{极限与连续性}
\begin{block}{定义1.3.3}
 $ f(z) $ 在 $ E $ 上有定义, $ z_0 $ 是 $ E $ 的聚点, $ z\in E $. 如果
\[\lim\limits_{z \to {z_0}} f ( z ) = f ( {z_0} ) \]
称 $ f(z) $ 在 $ z_0 $ 连续. 
\bigskip


即,  $ \forall\: \varepsilon >0, \exists\: \delta (\varepsilon, z_0)>0 $,  只要 $ |z- z_0|<\delta $, 就有 $ | f(z) - f(z_0) |<\delta $. 

\bigskip

如果 $ E $ 内每一点都是聚点, $ f(z) $ 在 $ E $ 上每一点都连续, 称 $ f(z) $ 在 $ E $ 上连续. 
\end{block}

\end{frame}

\begin{frame}{极限与连续性}
\begin{block}{定理1.3.3}
\begin{enumerate}
	\item 在 $ z_0 $ 连续的两个函数 $ f(z) $ 和 $ g(z) $ 的和、差、积、商(分母在 $ z_0 $ 不为零)在 $ z_0 $ 处仍连续.
	\item 设函数 $ h=g(z) $ 在 $ z_0 $ 连续, 函数 $ w=f(h) $ 在 $ h_0=g(z_0) $ 连续, 那么复合函数 $ w=f [g(z)] $ 在 $ z_0 $ 处连续.
\end{enumerate}
\end{block}


\end{frame}

\begin{frame}{极限与连续性}

\begin{block}{定理1.3.4}
	设 $ f(z)=u(x, y)+iv(x, y) $, 则函数 $ f(z) $ 在 $ z_0=x_0+iy_0 $ 连续的充分必要条件是 $ u(x, y) $ 和 $ v(x, y) $ 在 $ (x_0, y_0) $ 处连续
	
	举例说明如下: 
	
	\[ f(z)=\ln(x^2+y^2)+i(x^2-y^2) \]
	
	$ u(x, y) = \ln(x^2+y^2) $ 在复平面内除原点外处处连续, $ v(x, y)=i(x^2-y^2) $ 在复平面内处处连续, 因此, $ f(z) $ 在复平面内除原点外处处连续 $ f(z) $ 表示的函数是什么? 
	
\end{block}


该定理将复变函数 $ f(z) = u(x, y) + iv(x, y) $ 的连续性问题与两个二元实函数 $ u(x, y) $ 和 $ v(x, y) $ 的连续性问题密切联系在一起.
\end{frame}

\begin{frame}{极限与连续性}
\begin{block}{例2}
证明: 如果 $ f(z) $ 在 $ {z_0} $ 连续, 那末 $ \overline {f(z)} $ 在点 $ {z_0} $ 处也连续.

\begin{proof}
设 $ f(z) = u(x, y) + iv(x, y) $,  

则 $ \overline {f(z)} = u(x, y) - iv(x, y) $, 

由 $ f(z) $ 在 $ {z_0} $ 连续, 

知 $ u(x, y) $ 和 $ v(x, y) $ 在 $ ({x_0}, {y_0}) $ 处都连续, 

于是 $ u(x, y) $ 和 $ - v(x, y) $ 也在 $ ({x_0}, {y_0}) $ 处连续, 

故 $ \overline {f(z)} $ 在 $ {z_0} $ 连续.
\end{proof}
\end{block}
\end{frame}

\begin{frame}{极限与连续性}
(2) 一致连续

\begin{block}{定义}
 $ \forall\: \varepsilon >0, \exists\: \delta(\varepsilon) >0  $,  只要 $ z_1, z_2 \in E $, 且 $ |z_1- z_2|<\delta $, 就有 $ | f(z_1)- f(z_2) |< \varepsilon $ 
\end{block}



\end{frame}


\begin{frame}{极限与连续性}
(3) 有界闭域 $ \bar{D} $ 上连续函数 $ f(z) $ 的性质

a. $ |f(z)| $ 在 $ \bar{D} $ 上有界, 并达到其最小上界和最大下界. 

\begin{proof}
	1) 如果 $ |f(z)| $ 无界, 则有 $ \{z_n\}\in \bar{D} $, 满足: 
	\begin{equation}\label{key}
	|f(z_n)|>n, n=1, 2, \dots
	\end{equation}
	
	但 $ \{z_n\} $ 有界, 设 $ z_0 $ 是 $ \{z_n\} $ 的一个聚点, 则 $ z_0 \in \bar{D} $, 有 $ \{z_{nk}\} \subset \{z_n\}, k \to \infty, z_{nk} \to z_0 $. 由 $ f(z) $ 在 $ z_0 $ 连续, 得
	\[\lim\limits_{{\rm{k}} \to \infty } |f ( {z_{{\rm{nk}}}} ) | = |f ( {z_0} ) |\]
	即 $ f(z_{nk}) $ 有界, 这与(I)矛盾. 
\end{proof}
\end{frame}

\begin{frame}{极限与连续性}
\begin{proof}[]
2) 设 $ m $ 和 $ M $ 分别是 $ |f(z)| $ 在 $ \bar{D} $ 上的最大下界和最小上界, 则对于 $ 1/n>0 $, 有 $ \{z_n\} \in \bar{D} $,  $ \{z'_n\} \in \bar{D} $,  满足: 
\begin{align*}
&M \geq |f(z_n)| > M-1/n
&m \leq |f(z'_n)| < m+1/n
\end{align*}

设 $ z_0 $ 是 $ \{z_n\} $ 的聚点,  $ z'_0 $ 是 $ \{z'_n\} $ 的聚点, 则 $ z_0, z'_0 \in \bar{D} $,  有 $ \{z_{nk}\} \subset \{z_n\}, \{z'_{nk}\} \subset \{z'_n\}, k \to \infty, z_{nk} \to z_0, z'_{nk} \to z'_0 $,  由于 $ f(z) $ 的连续性, 使得 
\begin{align*}
&|f ( {z_0} ) | = \lim\limits_{{k} \to \infty } |f ( {z_{nk}} ) | = M\\
&|f ( z'_{0} ) | = \lim\limits_{{k} \to \infty } |f ( z'_{{nk}} ) | = {{m}}
\end{align*}
\end{proof}

\end{frame}

\begin{frame}{极限与连续性}
b. 闭域  $ \bar{D} $  上的连续函数 $ f(z) $ 必一致连续. 
\begin{proof}
反证法. 设 $ f(z) $ 在 $ \bar{D} $  上不一致连续, 即 $ \exists\: \varepsilon >0 $, 对于 $ \delta_n = 1/n $,  $ \exists\: z_n, z'_n \in \bar{D} $,  使得 $ |z_n - z'_n|< 1/n $,  而 $ |f(z_n) - f(z'_n) | > \varepsilon $, 对所有的 $ n $ 成立. 

设 $ z_0 $ 是 $ \{z_n\} $ 的聚点, 则 $ z_0 \in \bar{D} $, 有 $ \{z_{nk}\} \subset \{z_n\} $, 使得
$ \lim\limits_{k \to \infty } {z_{nk} = {z}_{0}} $,  而 
\begin{align*}
|z'_{nk}- z_0 |& \leq |z'_{nk}- z_{nk} | + |z_{nk} - z_0 |\\
&< 1/n + |z_{nk} - z_0 | \to 0 (k \to \infty)
\end{align*}

即 $ \lim\limits_{{k} \to \infty } z'_{nk}= {z}_{0} $ 
这表明,  $ z_0 $ 是 $ \{z_{nk}\}, \{z'_{nk}\} $ 的公共聚点. 

由 $ |f(z_n) - f(z'_{n}) |>\varepsilon $,  两边取极限, 根据 $ f(z) $ 的连续性 $ \varepsilon \leq |f(z_0) - f(z_0) |=0 $, 与 $ \varepsilon >0 $ 矛盾. 
\end{proof}

\end{frame}


\begin{frame}{本章主要内容(一)}
\begin{figure}
\includegraphics[height=0.8\textheight]{104}
\end{figure}
\end{frame}

\begin{frame}{本章主要内容(二)}
\begin{figure}
	\includegraphics[width=0.8\linewidth]{105}
\end{figure}
\end{frame}

\begin{frame}{本章要注意的几点}
\begin{itemize}
	\item 复数运算和各种表示法
	\item 复数方程表示曲线以及不等式表示区域
	
	\item 聚点、内点、开集、区域
	
	\item 可求长曲线、简单曲线
	
	\item 序列聚点、B-W定理、柯西序列
	
	\item 一致连续
	
\end{itemize}
\end{frame}

\begin{frame}{作业}
\huge

书P25 -- 27

\bigskip

3、5(4)、9、 10、14(4)、16(8)、17、 20、25、27.

\end{frame}

\begin{frame}{}

\Huge
\centering
\vfill
第一章\qquad 完
\vfill

\end{frame}


\begin{frame}{L. Euler(欧拉)简介}

\begin{minipage}[t]{0.3\linewidth}
\begin{figure}\vspace{-3mm}
\includegraphics[width=\linewidth]{109}
\end{figure}
\end{minipage}\qquad
\begin{minipage}[t]{0.6\linewidth}
1707.4.15 生于瑞  士, 巴塞尔

1783.9.18卒于俄罗斯, 彼得堡
\bigskip

Euler是18世纪的数学巨星; 是那个时代的巨人, 科学界的代表人物. 历史上几乎可与Archimedes、Newton、Gauss齐名. 
\end{minipage}
\bigskip

他在微积分、几何、数论、变分学等领域有巨大贡献. 可以说 Newton、Leibniz发明了微积分, 而Euler则是数学大厦的主要建筑师. 

\end{frame}
\begin{frame}{A. de Moivre 棣莫佛简介}

\begin{minipage}[t]{0.3\linewidth}
	\begin{figure}\vspace{-3mm}
		\includegraphics[width=\linewidth]{110}
	\end{figure}
\end{minipage}\qquad
\begin{minipage}[t]{0.6\linewidth}
1667.5.26 生于法国

1754.11.27 卒于英国
\bigskip

在概率论、复数理论等领域做了一些出色的工作. 
\bigskip

解决斐波那契数列的通项问题. L.Fibonacci (1170--1250)
\end{minipage}


\end{frame}

% \begin{frame}[t, fragile]{视频嵌入的例子}

% %\movie[externalviewer,label=mymovie,width=1in,height=0.8in,poster]{\pgfuseimage{01}}{HowtoDraw01.mp4}
% \begin{figure}[ht]{fragile}
%       \includemovie[
%         poster,
%         text={\small(Loading HowtoDraw01.mp4)}
%       ]{6cm}{6cm}{HowtoDraw01.mp4}
%       \end{figure}

% \end{frame}

\subsection{复变函数}

%------------------------------------------------
\section{第二章 \quad 解析函数}

\section{第三章 \quad  复变函数的积分}

\section{第四章 \quad 留数}

\section{第五章 \quad 级数}

\section{第六章 \quad 卷积积分与积分变换}

%----------------------------------------------------------------------------------------

\end{document}
